\documentclass{article}

% set font encoding for PDFLaTeX or XeLaTeX
\usepackage{ifxetex}
\ifxetex
  \usepackage{fontspec}
\else
  \usepackage[T1]{fontenc}
  \usepackage[utf8]{inputenc}
  \usepackage{lmodern}
\fi

% used in maketitle
\title{Actividad 2}
\author{Jennifer Judith Salgado Parra \\ Universidad de Sonora\\ Departamento de Física \\}
\date {Septiembre 11, 2017}

\begin{document}
\maketitle
\section{Movimiento de Proyectiles}

\subsection {Definición}
El movimiento del proyectil es una forma de movimiento en la que un objeto o partícula (en cualquier caso denominado proyectil ) se lanza cerca de la superficie de la Tierra, y se mueve a lo largo de un camino curvo bajo la acción de la gravedad solamente, donde la resistencia del aire no se toma en cuenta.

\subsection{La velocidad inicial}
Ésta podemos expresarla como la suma de componentes horizontales y verticales, de esta manera:\[v_{0}=v_{0x}i + v_{0y}j\]
En caso que queramos encontrar las componentes, es necesario contar con el ángulo inicial \(\theta\) y la velocidad inicial \(v_{0}\), para así utilizar las siguientes fórmulas. \[v_{0x}=v_{0}cos\theta\]\[v_{0y}=v_{0}sen\theta\]

En caso que necesitemos encontrar la velocidad inicial, teniendo la posición del proyectil (x,y) y el ángulo de lanzamiento \(\theta\), se utilizará la siguiente fórmula:\[v_{0}=\sqrt{\frac{x^2g}{xSen(2\theta)-2yCos^2(\theta)}}\]

\subsection{Aceleración}
Dado que sólo hay aceleración en la dirección vertical (trarándose de un movimiento en caída libre); la velocidad en la dirección horizontal es constante. Siendo así: \[a_{x}=0\]\[a_{y}=-g\]

\subsection{Velocidad}
La magnitud de la velocidad dada por el teorema de pitágoras se definirá de la siguiente forma:
\[v={\sqrt{v^2_{x}+v^2_{y}}}\]

\subsection{Desplazamiento}
En cualquier momento t, el desplazamiento horizontal y vertical del proyectil son:\[x=v_{0} t cos(\theta)\]
\[y=v_{0} t sen(\theta)-\frac{1}{2}gt^2 \]

\subsection{Tiempo de vuelo}
El tiempo total t para el cual el proyectil permanece en el aire se llama el tiempo de  vuelo y está determinado con la siguiente fórmula \[t=\frac{2v_{0}sen(\theta)}{g}\]

\subsubsection{Ejemplo} Un hombre lanza una pelota con un ángulo de \(\theta=25\)$^{\circ}$ a una rapidez de 8 m/s. ¿Cuánto tiempo permanece la pelota en el aire?\\\\
Sustituimos en la ecuación \[t=\frac{2\cdot 8 sen(20)}{9.8}\approx 0.558 s\]

\subsection{Altura máxima del proyectil}
La altura máxima que el objeto alcanzará será cuando \(v_{y}=0\) y está dada por la siguiente fórmula
\[h=\frac{v^2_{0}sen^2(\theta)}{2g}\]
Por otra parte, si queremos encontrar el tiempo para alcanzar la altura máxima, basta con igualar a cero la ecuación del desplazamiento en \textit{y}, y despejar \textit{t}, quedándonos así\[t_{h}=\frac{v_{0}sen(\theta)}{g}\]

\subsubsection{Ejemplo} Un hombre lanza una pelota con un ángulo de \(\theta=25\)$^{\circ}$ a una rapidez de 8 m/s. ¿En qué altura su velocidad es cero?\\\\
Sustituimos en la ecuación \[h=\frac{8^2 \cdot sen^2(25)}{2 \cdot 9.8} \approx 0.381 m\]

\subsection{Distancia máxima del proyectil}
Hemos de notar que para encontrar el rango horizontal del proyectil es cuando la altura en \textit{y} vuelve a ser cero. Por lo tanto, igualamos a cero la ecuación en desplazamiento en \textit{y}.
\[0=v_{0} t_{d} sen(\theta)-\frac{1}{2}gt^2_{d}\]
Despejamos \textit{\(t_{d}\)}, tiempo en el que se realizó la distancia máxima recorrida del proyectil. \[t_{d}=\frac{v_{0}sen(\theta)}{g}\]
Teniendo el tiempo en el que realizó el proyectil todo el recorrido, sustituimos en la ecuación de desplazamiento en \textit{x}.
\[d=v_{0}t_{d}cos(\theta)\]
Utilizando la identidad trigonométrica \textit{\(2\cdot sen(\theta)\cdot cos(\theta)=sen(2\theta)\)} nos queda que:
\[d=\frac{v^2_{0}sen(2\theta)}{g}\]

\subsubsection{Ejemplo} Un hombre lanza una pelota con un ángulo de \(\theta=25\)$^{\circ}$ a una rapidez de 8 m/s. ¿Qué distancia recorre la pelota?\\\\
Sustituimos en la ecuación \[d=\frac{8^2 \cdot sen(2 \cdot 20)}{9.8} \approx 4.197 m\]

\section{Bibliografía}
\begin{thebibliography}{a}

\bibitem{wiki} \textsc{Wikipedia.}(Agosto 31, 2017). Movimiento de proyectiles.2017, de Fundación Wikimedia, Inc. Sitio web: \textit{https://en.wikipedia.org/wiki/Projectile\_motion}

\end{thebibliography}


\end{document}
